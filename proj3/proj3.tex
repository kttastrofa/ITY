%preambula
\documentclass[11pt]{article}

%packages
\usepackage{geometry}
\usepackage{babel}[czech]
\usepackage{times}
\usepackage{picture}
\usepackage{tabbing}
\usepackage{tabular}
\usepackage{graphicx}
\usepackage[hidelinks]{hyperref}



%pack++
\geometry{a4paper,
          total={17cm, 24cm},
          left=2cm,
          top=3cm,}
\hypersetup{colorlinks=true,
            linkcolor=black,
            filecolor=black,
            urlcolor=black,}


%document
\begin{document}
    %titulna strana
    \begin{titlepage}
        \begin{center}
            \Huge\textsc{Vysoké učení technické v Brně}\\
            \huge\textsc{Fakulta informačních technologií}\\
            \vspace{\stretch{0.382}}
            {Typografie a publikování –– 3. projekt}\\
            \Huge{Tabulky a obrázky}\\
            \vspace{\stretch{0.618}}
        \end{center}
        {\LARGE 27. března 2024 \hfill
        Katarína Mečiarová (xmeciak00)}

    \end{titlepage}

%torzo
    \section{Úvodní strana}
        Název práce umístěte do zlatého řezu a nezapomeňte uvést dnešní datum a vaše jméno a příjmení.

    \section{Tabulky}
        Pro sázení tabulek můžeme použít bud prostředí \verb|tabbing| nebo prostředí \verb|tabular|.

        \subsection{Prostředí \texttt{tabbing}}
            Při použití \verb|tabbing| vypadá tabulka následovně:
            \begin{tabbing}
                    Ovoce           \= Cena     \= Množství     \\
                    Jablka          \> 25,90    \> 3 kg         \\
                    Hrušky          \> 27,40    \> 2,5 kg       \\
                    Vodní melouny   \> 35,-     \> 1 kus        \\
            \end{tabbing}
            %\\
            Toto prostředí se dá také použít pro sázení algoritmů, ovšem vhodnější je použít prostředí \verb|algorithm| nebo \verb|algorithm2e| (viz sekce 3).

        \subsection{Prostředí \texttt{tabular}}
            Další možností, jak vytvořit tabulku, je použít prostředí \verb|tabular|. Tabulky pak budou vypadat takto\footnote[1]{Kdyby byl problem s \verb|cline|, zkuste se podívat třeba sem: \href{http://www.abclinuxu.cz/tex/poradna/show/325037}{http://www.abclinuxu.cz/tex/poradna/show/325037}.}:

            \begin{center}
                \begin{tabular}{|l|c|c|}
                    \hline
                    \multirow{2}{*}{Měna} & \multicolumn{2}{|c|}{Cena} \\

                                            \cline { 2 - 3 }
                                            & nákup     & prodej \\

                    \hline
                                EUR         & 25,475    & 27,045    \\
                                GBP         & 28,835    & 30,705    \\
                                USD         & 22,943    & 24,357    \\
                    \hline
                \end{tabular}
            \end{center}


            Tabulka 1: Tabulka kurzů $\mathrm{k}$ dnešnímu dni
        %todo bordel upravit do poradku = 3 tabulky, podla [] zarovnane do riadku, nem 1 velka lol
            \begin{center}
                \begin{tabular}{|c|c|c|c|c|c|c|c|c|c|c|c|c|c|c|c|c|c|c|c|}
                    \hline
                    &  & \multicolumn{2}{|c|}{\multirow{2}{*}}{$A \wedge B$} & \multicolumn{4}{|c|}{$B$} & \multicolumn{2}{|c|}{\multirow{2}{*}}{$A \vee B$} & \multicolumn{4}{|c|}{$B$} & \multicolumn{2}{|c|}{\multirow{2}{*}}{$A \rightarrow B$} & \multicolumn{4}{|c|}{$B$} \\
                    \hline
                    $\frac{A}{\mathbf{P}}$ & $\frac{\neg A}{\mathrm{~N}}$ &  &  & $\mathbf{P}$ & O & $\mathbf{X}$ & $\mathbf{N}$ &  &  & $\mathbf{P}$ & $\mathbf{O}$ & $\mathbf{X}$ & $\mathbf{N}$ &  &  & $\mathbf{P}$ & O & $\mathbf{X}$ & $\mathbf{N}$ \\
                    \hline
                    $\frac{1}{0}$ & $\frac{\mathrm{N}}{\mathrm{O}}$ &  & $\mathbf{P}$ & $\mathrm{P}$ & $\mathrm{O}$ & $\mathrm{X}$ & $\mathrm{N}$ &  & $\mathbf{P}$ & $\mathrm{P}$ & $\mathrm{P}$ & $\mathrm{P}$ & $\mathrm{P}$ &  & $\mathbf{P}$ & $\mathrm{P}$ & $\mathrm{O}$ & $\mathrm{X}$ & $\mathrm{N}$ \\
                    \hline
                    $\frac{U}{Y}$ & $\frac{\mathrm{U}}{\mathrm{Y}}$ & $\Delta$ & 0 & $\mathrm{O}$ & $\mathrm{O}$ & $\bar{N}$ & $\mathrm{~N}$ & 4 & 0 & $\mathrm{P}$ & $\mathrm{O}$ & $\mathrm{P}$ & $\mathrm{O}$ & 4 & $\mathbf{O}$ & $\bar{P}$ & $\mathrm{O}$ & $\mathrm{P}$ & $\overline{\mathrm{O}}$ \\
                    \hline
                    $\mathbf{A}$ & $\frac{A}{D}$ & A & $\mathbf{X}$ & $\mathrm{X}$ & $\mathrm{N}$ & $\mathrm{X}$ & $\mathrm{N}$ &  & $\mathbf{X}$ & $\mathrm{P}$ & $\mathrm{P}$ & $X$ & $\mathrm{X}$ & $A$ & $\mathbf{X}$ & $\mathrm{P}$ & $\mathrm{P}$ & $\mathrm{X}$ & $\bar{X}$ \\
                    \hline
                    N & $P$ &  & $\mathbf{N}$ & $\bar{N}$ & $\mathrm{~N}$ & $\bar{N}$ & $\mathrm{~N}$ &  & $\mathbf{N}$ & $\mathrm{P}$ & $\mathrm{O}$ & $X$ & $\mathrm{~N}$ &  & $\mathbf{N}$ & $\mathrm{P}$ & $\mathrm{P}$ & $\mathrm{P}$ & $\mathrm{P}$ \\
                    \hline
                \end{tabular}
            \end{center}

            Tabulka 2: Protože Kleeneho trojhodnotová logika už je „zastaralá“, uvádíme si zde příklad čtyřhodnotové logiky


    \section{Algoritmy}
        Pokud budeme chtít vysázet algoritmus, můžeme použít prostředí \verb|algorithm|\footnote[2]{Pro nápovědu, jak zacházet s prostředím \verb|algorithm|, mužeme zkusit tuhle stránku:\href{http://ftp.cstug.cz/pub/tex/CTAN/macros/latex/contrib/algorithms/algorithms.pdf}{http://ftp.cstug.cz/pub/tex/CTAN/macros/latex/contrib/algorithms/algorithms.pdf}.} nebo \verb|algorithm2|\footnote[3]{Pro \verb|algorithm2| zase tuhle: \href{http://ftp.cstug.cz/pub/tex/CTAN/macros/latex/contrib/algorithm2e/doc/algorithm2e.pdf}{http://ftp.cstug.cz/pub/tex/CTAN/macros/latex/contrib/algorithm2e/doc/algorithm2e.pdf}.} Příklad použití prostředí \verb|algorithm2e| viz Algoritmus 1.


    \section{Obrázky}



\end{document}