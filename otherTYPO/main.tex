% zde budiž můj tahák na ima1
% dávám k dispozici latex abyste si to mohli
% upravit/vykopírovat-vykrást nebo whatever
% renderuju to na overleaf ofc
% autor: vít pavlík
% datum: 10. 6. 2023

% tady je vhodne 14 pt na zobrazovani elektronicky, 
% na vytisknuti na realnou A4 asi staci 10, 11 nebo 12pt (da se i 8pt)
% jde jen 8pt, 9pt, 10pt, 11pt, 12pt, 14pt, 17pt, 20pt
\documentclass[8pt, a5paper]{extarticle}
\usepackage{pgfpages}
\usepackage{atbegshi}
\pgfpagesuselayout{4 on 1}[a4paper]

\pgfpageslogicalpageoptions{1}{copy from=1}
\pgfpageslogicalpageoptions{2}{copy from=1}
\pgfpageslogicalpageoptions{3}{copy from=1}
\pgfpageslogicalpageoptions{4}{copy from=1}
\pgfpageslogicalpageoptions{5}{copy from=2}
\pgfpageslogicalpageoptions{6}{copy from=2}
\pgfpageslogicalpageoptions{7}{copy from=2}
\pgfpageslogicalpageoptions{8}{copy from=2}


\AtBeginShipout{%
  \pgfpagesshipoutlogicalpage{1}\copy\AtBeginShipoutBox
  \pgfpagesshipoutlogicalpage{2}\copy\AtBeginShipoutBox
  \pgfpagesshipoutlogicalpage{3}\copy\AtBeginShipoutBox
  \pgfpagesshipoutlogicalpage{4}\copy\AtBeginShipoutBox
  \pgfpagesshipoutlogicalpage{5}\copy\AtBeginShipoutBox
  \pgfpagesshipoutlogicalpage{6}\copy\AtBeginShipoutBox
  \pgfpagesshipoutlogicalpage{7}\copy\AtBeginShipoutBox
  \pgfpagesshipoutlogicalpage{8}\copy\AtBeginShipoutBox
  \pgfshipoutphysicalpage
}



\usepackage[fleqn]{amsmath}

% kvuli datu aby tam nebylo
\date{\vspace{-5ex}}
%\date{}  % Toggle commenting to test

% Language setting
% Replace `english' with e.g. `spanish' to change the document language
\usepackage[czech]{babel}

% Set page size and margins
% Replace `letterpaper' with `a4paper' for UK/EU standard size
%\usepackage[]{geometry}

% Useful packages
\usepackage{amsmath}
\usepackage{graphicx}
\usepackage[colorlinks=true, allcolors=blue]{hyperref}
\usepackage{setspace}
\usepackage{multicol}


\begin{document}
  \setstretch{0.8}
  \begin{multicols}{2}

    \[
      sin^2(x) + cos^2(x) = 1
    \]
    \[
      tg(x) \cdot cotg(x) = 1
    \]
    \[
      tg(x) = \frac{sin(x)}{cos(x)}
    \]
    \[
      cotg(x) = \frac{cos(x)}{sin(x)}
    \]
    \[
      sin(2x) = 2sin(x)cos(x)
    \]
    \[
      cos(2x) = cos^2(x)-sin^2(x)
    \]
    \[
      sin(a + b) = sin(a)cos(b) + cos(a)sin(b)
    \]
    \[
      sin(a - b) = sin(a)cos(b) - cos(a)sin(b)
    \]
    \[
      cos(a + b) = cos(a)cos(b) - sin(a)sin(b)
    \]
    \[
      cos(a - b) = cos(a)cos(b) + sin(a)sin(b)
    \]
    \[
      \int \frac{f'(x)}{f(x)}dx = ln \left| f(x) \right| + C
    \]
    \[ \int f(ax + b)dx = \frac{1}{a} F(ax + b) + C, \; F(x) = \int f(x)\]
    \[
      \int k dx = kx + C
    \]
    \[
      \int x^n dx = \frac{x^{n+1}}{n+1} + C \quad (n \neq -1)
    \]
    \[
      \int \frac{1}{x} dx = \ln|x| + C
    \]
    \[
      \int e^x dx = e^x + C
    \]
    \[
      \int a^x dx = \frac{a^x}{\ln(a)} + C
    \]
    \[
      \int \sin (x) dx = -\cos x + C
    \]
    \[
      \int \cos (x) dx = \sin x + C
    \]
    \[
      \int \tan (x) dx = -\ln|\cos(x)| + C
    \]
    \[
      \int \frac{1}{a^2+x^2}dx = \frac{1}{a} \arctan\left(\frac{x}{a}\right) + C
    \]
    \[
      \int \frac{1}{\sqrt{a^2-x^2}}dx = \arcsin\left(\frac{x}{a}\right) + C
    \]
    \[
      \int \frac{1}{\sin^2(x)} = \cot(x) + C
    \]
    \[
      \int \frac{1}{\cos^2(x)} = \tan(x) + C
    \]
    \[
      \int \sin^2(x) dx = \frac{x - \sin(x)\cos(x)}{2} + C
    \]
    \[\int \ln(x) dx = x(\ln(x)-1) + C\]
    \begin{itemize}
      \item to by se spočítalo per partes, kde $u=\ln(x); v' = 1$
    \end{itemize}
  \end{multicols}
  \textbf{Asymptota se směrnicí}
  \[
    y = ax + b
  \]
  \[
    a=\lim_{x\to\infty} \frac{f(x)}{x} \\
  \]
  \[
    b=\lim_{x\to\infty} f(x)-kx
  \]
  \textbf{Objem rotačního tělesa}
  \[
    V = \pi \cdot \int_{a}^{b}(f(x))^2 \: dx \;\; j^2
  \]
  \textbf{Simpsonovo pravidlo 1/3 (kvadratická interpolace)}
  \[
    \int_{a}^{b} f(x)dx = \frac{b-a}{6} \left( f(a) + 4f\left(\frac{a+b}{2}\right) + f\left(b\right) \right)
  \]

  \textbf{Elipsa}
  \begin{multicols}{2}
    \[
      \frac{(x-m)^2}{a^2} + \frac{(y-n)^2}{b^2} = k
    \]

    kde \textbf{a} a \textbf{b} jsou délky hlavní a vedlejší poloosy, \textbf{m} a \textbf{n} je posunutí a \textbf{k} je součinitel velikosti (obyčejně v té rovnici bývá jednička ale o elipsu se jedná i když je to jiné číslo)
  \end{multicols}
  \begin{multicols}{3}
    \begin{tabular}{|c|c|c|c|}
      \hline
      $\theta$ & $\theta$ & $\sin(\theta)$ & $\cos(\theta)$ \\
      \hline
      $0^\circ$ & $0$ & $0$ & $1$ \\
      $30^\circ$ & $\frac{\pi}{6}$ & $\frac{1}{2}$ & $\frac{\sqrt{3}}{2}$ \\
      $45^\circ$ & $\frac{\pi}{4}$ & $\frac{\sqrt{2}}{2}$ & $\frac{\sqrt{2}}{2}$ \\
      $60^\circ$ & $\frac{\pi}{3}$ & $\frac{\sqrt{3}}{2}$ & $\frac{1}{2}$ \\
      $90^\circ$ & $\frac{\pi}{2}$ & $1$ & $0$ \\
      \hline
    \end{tabular}

    \[
      (fg)' = f'g \cdot fg'
    \]
    \[
      \left ( \frac{f}{g} \right )' = \frac{f'g - fg'}{g^2}
    \]
    \[
      (f(g))' = f'(g) \cdot g'
    \]
    \[
      (ln(x))'=\frac{1}{x}
    \]
  \end{multicols}
  \begin{multicols}{2}
    \section{newtonova metoda}
    podminky konvergence:

    1: spojita prvni i druha derivace

    2: prvni derivace nikde neni nulova

    3: prvni ani druha derivace nemeni znamenko

    volime $x_0$ tak aby funkcni hodnota a druha derivace meli stejne znamenko

    vzorec pro dalsi aproximace
    \[
      x_{n+1}=x_n - \frac{f(x_n)}{f'(x_n)}\]
  \end{multicols}
  \begin{multicols}{5}
  [
  ]
  {\footnotesize{{
    \[
          \sqrt{1} = 1,00
        \]
        \[
          \sqrt{2} = 1,41
        \]
        \[
          \sqrt{3} = 1,73
        \]
        \[
          \sqrt{4} = 2,00
        \]
        \[
          \sqrt{5} = 2,24
        \]
        \[
          \sqrt{6} = 2,45
        \]
        \[
          \sqrt{7} = 2,65
        \]
        \[
          \sqrt{8} = 2,83
        \]
        \[
          \sqrt{9} = 3,00
        \]
        \[
          \sqrt{10} = 3,16
        \]
        \[
          \sqrt{11} = 3,32
        \]
        \[
          \sqrt{12} = 3,46
        \]
        \[
          \sqrt{13} = 3,61
        \]
        \[
          \sqrt{14} = 3,74
        \]
        \[
          \sqrt{15} = 3,87
        \]
        \[
          \sqrt{16} = 4,00
        \]
        \[
          \sqrt{17} = 4,12
        \]
        \[
          \sqrt{18} = 4,24
        \]
        \[
          \sqrt{19} = 4,36
        \]
        \[
          \sqrt{20} = 4,47
        \]
        \[
          \sqrt{21} = 4,58
        \]
        \[
          \sqrt{22} = 4,69
        \]
        \[
          \sqrt{23} = 4,80
        \]
        \[
          \sqrt{24} = 4,90
        \]
        \[
          \sqrt{25} = 5,00
        \]
        \[
          \sqrt{26} = 5,10
        \]

        \[
          \sqrt[3]{1} = 1,00
        \]
        \[
          \sqrt[3]{2} = 1,26
        \]
        \[
          \sqrt[3]{3} = 1,44
        \]
        \[
          \sqrt[3]{4} = 1,59
        \]
        \[
          \sqrt[3]{5} = 1,71
        \]
        \[
          \sqrt[3]{6} = 1,82
        \]
        \[
          \sqrt[3]{7} = 1,91
        \]
        \[
          \sqrt[3]{8} = 2,00
        \]
        \[
          \sqrt[3]{9} = 2,08
        \]
        \[
          \sqrt[3]{10} = 2,15
        \]
        \[
          \sqrt[3]{11} = 2,22
        \]
        \[
          \sqrt[3]{12} = 2,29
        \]
        \[
          \sqrt[3]{13} = 2,35
        \]
        \[
          \sqrt[3]{14} = 2,41
        \]
        \[
          \sqrt[3]{15} = 2,47
        \]
        \[
          \sqrt[3]{16} = 2,52
        \]
        \[
          \sqrt[3]{17} = 2,57
        \]
        \[
          \sqrt[3]{18} = 2,62
        \]
        \[
          \sqrt[3]{19} = 2,67
        \]
        \[
          \sqrt[3]{20} = 2,71
        \]
        \[
          \sqrt[3]{21} = 2,76
        \]
        \[
          \sqrt[3]{22} = 2,80
        \]
        \[
          \sqrt[3]{23} = 2,84
        \]
        \[
          \sqrt[3]{24} = 2,88
        \]
        \[
          \sqrt[3]{25} = 2,92
        \]
        \[
          \sqrt[3]{26} = 2,96
        \]}}}

  \end{multicols}







\end{document}