% Preamble
\documentclass[a4paper, 10pt, twocolumn, article][18.3.2024]
\usepackage[top=1.8cm, inner=1.8cm, outer=1cm]{geometry}
\usepackage[utf8]{inputenc}
\usepackage[T1]{fontenc}
\usepackage[czech]{babel}

% Packages
\usepackage{alltt}
\usepackage{changepage}
\usepackage{enumitem}
\usepackage{setspace}
\usepackage{amsmath}


%document
\begin{document}
%title
\begin{titlepage}
    {Vysoké učení technické v Brně}
    Fakulta informačních technologií
        \title{Typografie a publikování –– 2. projekt}
    Sazba dokumentů a matematických výrazů
    2024 Katarína Mečiarová (xmeciak00)
\end{titlepage}

    \section{Úvod}
    V této úloze si vysázíme titulní stranu a kousek ma-
    tematického textu, v němž se vyskytují například De-
    finice 1 nebo rovnice (2) na straně 1. Pro vytvoření
    těchto odkazů používáme kombinace příkazů \label,
    \ref, \eqref a \pageref. Před odkazy patří nezlo-
    mitelná mezera. Pro zvýrazňování textu se používají
    příkazy \verb a \emph.
    Titulní strana je vysázena prostředím titlepage
    a nadpis je v optickém středu s využitím přesného zla-
    tého řezu, který byl probrán na přednášce. Dále jsou
    na titulní straně čtyři různé velikosti písma a mezi
    dvojicemi řádků textu je použito řádkování se zada-
    nou relativní velikostí 0,5 em a 0,6 em1.
    1 Matematický text
    Matematické symboly a výrazy v plynulém textu jsou
    v prostředí math. Definice a věty sázíme v prostředí
    definovaném příkazem \newtheorem z balíku amsthm.
    Tato prostředí obracejí význam \emph: uvnitř textu
    sázeného kurzívou se zvýrazňuje písmem v základ-
    ním řezu. Někdy je vhodné použít konstrukci ${}$
    nebo \mbox{}, která zabrání zalomení (matematic-
    kého) textu. Pozor také na tvar i sklon řeckých písmen:
    srovnejte \epsilon a \varepsilon, \Xi a \varXi.
    Definice 1. Konečný přepisovací stroj neboli Mea-
    lyho automat je definován jako uspořádaná pětice
    tvaru M = (Q, Σ, Γ, δ, q0), kde:
    • Q je konečná množina stavů,
    • Σ je konečná vstupní abeceda,
    • Γ je konečná výstupní abeceda,
    • δ : Q × Σ → Q × Γ je totální přechodová funkce,
    • q0 ∈ Q je počáteční stav.
    1.1 Podsekce s definicí
    Pomocí přechodové funkce δ zavedeme novou funkci δ∗
    pro překlad vstupních slov u ∈ Σ∗ do výstupních slov
    w ∈ Γ ∗.
    Definice 2. Nechť M = (Q, Σ, Γ, δ, q0) je Mealyho
    automat. Překládací funkce δ∗ : Q × Σ∗ × Γ ∗ → Γ ∗
    je pro každý stav q ∈ Q, symbol x ∈ Σ, slova u ∈ Σ∗,
    w ∈ Γ ∗ definována rekurentním předpisem:
    • δ∗(q, ε, w) = w
    • δ∗(q, xu, w) = δ∗(q′, u, wy), kde (q′, y) = δ(q, x)
    1Použijte správný typ mezery mezi číslem a jednotkou.
    1.2 Rovnice
    Složitější matematické formule sázíme mimo plynulý
    text pomocí prostředí displaymath. Lze umístit i více
    výrazů na jeden řádek, ale pak je třeba tyto vhodně
    oddělit, například pomocí \quad, při dostatku místa
    i \qquad.
    gan /∈ ABn
    y1
    0 − 5
    √
    x + 7
    √y x > y2 ≥ y3
    Velikost závorek a svislých čar je potřeba přizpů-
    sobit jejich obsahu. Velikost lze stanovit explicitně,
    anebo pomocí \left a \right. Kombinační čísla sá-
    zejte makrem \binom.
    ∣
    ∣
    ∣⋃ P
    ∣
    ∣
    ∣ = ∑
    ∅6=X⊆P
    (−1)|X|−1 ∣
    ∣
    ∣⋂ X
    ∣
    ∣
    ∣
    Fn+1 =
    (n
    0
    )
    +
    (n − 1
    1
    )
    +
    (n − 2
    2
    )
    + · · · +
    (⌈ n
    2
    ⌉
    ⌊ n
    2
    ⌋
    )
    V rovnici (1) jsou tři typy závorek s různou expli-
    citně definovanou velikostí. Obě rovnice mají svisle za-
    rovnaná rovnítka. Použijte k tomu vhodné prostředí.
    ({
        b ⊗ [c1 ⊕ c2
        ] ◦ a
    } 2
    3
    )
    = logz x (1)
    ∫ b
    a
    f (x) dx = −
    ∫ a
    b
    f (y) dy (2)
    V této větě vidíme, jak se vysází proměnná určující
    limitu v běžném textu: limm→∞ f (m). Podobně je to
    i s dalšími symboly jako ⋃
    N ∈M N či ∑m
    i=1 x2
    i . S vynu-
    cením méně úsporné sazby příkazem \limits budou
    vzorce vysázeny v podobě lim
    m→∞ f (m) a m∑
    i=1
    x2
    i .
    2 Matice
    Pro sázení matic se používá prostředí array a závorky
    s výškou nastavenou pomocí \left, \right.
    D =
    ∣
    ∣
    ∣
    ∣
    ∣
    ∣
    ∣
    ∣
    ∣
    ∣
    a11 a12 · · · a1n
    a21 a22 · · · a2n
    ... ... . . . ...
    am1 am2 · · · amn
    ∣
    ∣
    ∣
    ∣
    ∣
    ∣
    ∣
    ∣
    ∣
    ∣
    =
    ∣
    ∣
    ∣
    ∣
    ∣
    x y
    t w
    ∣
    ∣
    ∣
    ∣
    ∣ = xw − yt
    Prostředí array lze úspěšně využít i jinde, například
    na pravé straně následující rovnosti.
    (n
    k
    )
    =
        { n!
    k!(n−k)! pro 0 ≤ k ≤ n
    0 jinak
    1



\end{document}