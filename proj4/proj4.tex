%preambula
\documentclass[11pt]{article}

%packages
\usepackage[a4paper,
            total={17cm, 24cm},
            left=2cm,
            top=3cm,]{geometry}
\usepackage{times}
\usepackage[utf8]{inputenc}
\usepackage[T1]{fontenc}
\usepackage{tocloft}
\usepackage{bibentry}
\usepackage{indentfirst}
\usepackage{enumitem}
\PassOptionsToPackage{hyphens}{url}\usepackage[hidelinks]{hyperref}




%document
\begin{document}

    \begin{titlepage}
        \begin{center}
            \Huge\textsc{Vysoké učení technické v Brně}\\
            \huge\textsc{Fakulta informačních technologií}\\
            \vspace{\stretch{0.382}}
            \Large\textsc{Typografie a publikování – 4. projekt}\\
            \Huge{Chomsky and Typography}

            \vspace{\stretch{0.618}}
        \end{center}
        {\LARGE 17. dubna 2024 \hfill
        Katarína Mečiarová}

    \end{titlepage}

    \tableofcontents
    \renewcommand{\cftsecleader}{\cftdotfill{\cftdotsep}}
    \setlength{\cftbeforesecskip}{0.5em}

    \newpage

    \section{Introduction}
    In this article you will find out something about typography, it's history and usage.
    The author tried to use a lot of sources in hope of satisfying the reader. \\
    We will look on some problems in typography. What the reader might find interesting enough
    is the fact, that we do work with Noam Chomsky's works, looking on the typography
    from psychological type of view. \\ Enjoy!

    \section{History}
    Typography has a captivating history that spans millennia\cite{ModernTypo}.
    From ancient Mesopotamia to the Gutenberg revolution, its evolution is marked by significant milestones.
    Johannes Gutenberg's invention of movable type in the 15th century ignited a printing revolution in Europe.
    The Renaissance saw typographic experimentation flourish\cite{rhythm}
    alongside artistic and cultural renaissance\ldots etc.

    Cultural shifts, such as the Renaissance, influenced typographic styles and aesthetics\cite{rhythm}.
    Or the Industrial Revolution's mass production spurred typographic innovation.

    Chomsky's theory revolutionized linguistics by proposing that humans possess an innate linguistic
    capacity. This idea of innate structures influencing language acquisition parallels the influence
    of cultural and technological shifts on typography.\cite{ModernTypo}
    Just as language acquisition is influenced by
    innate cognitive structures, typography is influenced by underlying cultural and technological factors.\cite{LR}

    \subsection{Usage}
    Typography plays a pivotal role in branding and visual communication strategies.\cite{LR}
    Typeface classification systems help designers select appropriate fonts for various contexts.
    Typographic hierarchy guides readers' attention and organizes information effectively.\cite{hmota}
    Custom fonts are utilized by companies to establish distinctive brand identities.\cite{types}
    Responsive typography ensures optimal readability across different devices and screen sizes.\cite{typo}
    Typography serves as both a practical communication tool and a form of artistic expression in modern design.\cite{emotions}
    \subsection{Chomsky's part}
    Chomsky's thesis emphasizes the universality of linguistic structures and the human capacity to generate
    infinite expressions from finite linguistic elements. Similarly, in typography, designers utilize typographic
    elements such as fonts, sizes, and layouts to generate an infinite variety of visual expressions from a finite
    set of typographic elements.\cite{culturalenviro}
    This parallels the idea of generative grammar, where a finite set of rules generates
    an infinite array of grammatically correct sentences.\cite{mag}
    Moreover, just as Chomsky's theory underscores the importance
    of understanding language structure for effective communication, typography relies on principles of typographic
    hierarchy and organization to convey meaning and guide reader comprehension.\cite{LR}

    \section{Conclusion}
    In conclusion, Chomsky's thesis on language acquisition and generative grammar offers insights into the influences
    shaping typography and its usage in visual communication. Just as linguistic structures are shaped by innate
    cognitive capacities and cultural influences, typography is influenced by cultural, historical, and technological
    factors.\cite{critisism}
    Moreover, the principles of generative grammar parallel the creative potential of typography,
    where designers generate meaning and visual expression through the strategic arrangement of typographic elements.
    By understanding these parallels, designers can leverage typographic principles to effectively communicate
    and engage with audiences in the digital age.



    \bibliographystyle{czechiso} % Specify the bibliography style
    \bibliography{proj4} % Specify the .bib file containing your references



\end{document}