% Preamble
\documentclass[a4paper, 10pt, twocolumn]{article}[25.2.2024]

% Packages
\usepackage[top=1.8cm, inner=1.8cm, outer=1cm]{geometry}
\usepackage[utf8]{inputenc}
\usepackage[T1]{fontenc}
\usepackage[czech]{babel}
\usepackage{alltt}
\usepackage{color}
\usepackage{changepage}
\usepackage{enumitem}
\usepackage{setspace}
\usepackage[hidelinks]{hyperref}


%title
\title{Typografie a publikování -- 1. projekt}
\author{Katarína Mečiarová\\ \href{mailto: xmeciak00@stud.fit.vutbr.cz}{xmeciak00@stud.fit.vutbr.cz}}
\date{ }


%document
\begin{document}

    \maketitle


    \section{Hladká sazba}
        Hladká sazba používá jeden stupeň, druh a řez písma.
        Sází se na stránku s pevně stanovenou šířkou.
        Skládá se z odstavců. Odstavec končí východovou řádkou.
        Věty nesmějí začínat číslicí.

        Zvýraznění barvou, podtržením, ani změnou písma se v~odstavcích nepoužívá.
        Hladká sazba je určena především pro delší texty, jako je beletrie.
        Porušení konzistence sazby působí v textu rušivě a unavuje čtenářův zrak.

    \section{Smíšená sazba}
        Smíšená sazba má o něco volnější pravidla.
        Klasická hladká sazba se doplňuje o další řezy písma pro zvýraznění důležitých pojmů.
        Existuje \uv{pravidlo}:

        \medskip
        \begin{adjustwidth}{0.75cm}{0.75cm}
        \hspace{0.3cm}
        Čím více \textsc{druhů}, \emph{řezů}, {\scriptsize velikostí}, \textcolor{red}{barev} písma a~jiných \texttt{efektů} použijeme, \underline{tím profesionálněji}
        bude {\fontfamily{pzc}\selectfont dokument} vypadat. Čtenář tím bude {\Large \bfseries vždy nadšen!}
        \end{adjustwidth}
        \medskip

        \textsc{Tímto pravidlem se nesmíte \underline{\textbf{nikdy}} řídit.}
        Příliš časté zvýrazňování textových elementů a změny {\tiny velikosti} písma jsou známkou \textbf{amatérismu} autora a působí \texttt{velmi rušivě}.
        Dobře navržený dokument nemá obsahovat více než 4 řezy či druhy písma.
        Dobře navržený dokument je \underline{decentní, ne chaotický.}

        Důležitým znakem správně vysázeného dokumentu je konzistence -- například \textbf{tučný řez} písma bude vyhrazen pouze pro klíčová slova, \textit{kurzíva} pouze pro doposud neznámé pojmy a nebude se to míchat.
        Kurzíva nepůsobí tak rušivě a používá se častěji.
        V \LaTeX u ji sázíme raději příkazem \verb|\emph{text}| než \verb|\textit{text}|.

        Smíšená sazba se nejčastěji používá pro sazbu vědeckých článků a technických zpráv.
        U delších dokumentů vědeckého či technického charakteru je zvykem vysvětlit význam různých typů zvýraznění v úvodní kapitole.


    \section{Další rady:}
        \begin{itemize}
            \item Nadpis nesmí končit dvojtečkou a nesmí obsahovat odkazy na obrázky, citace, poznámky pod čarou, ...
            \item Nadpisy, číslování a odkazy na číslované elementy musí být sázeny příkazy k tomu určenými. Číslování sekcí tohoto dokumentu je zajištěno příkazem \verb|\section|.

            \item Poznámky pod čarou\footnote[1]{Příliš mnoho poznámek pod čarou čtenáře zbytečně rozptyluje.} používejte opravdu střídmě. (Šetřete i s textem v závorkách.)

            \item Bezchybným pravopisem a sazbou dáváme najevo úctu ke čtenáři. Odbytý text s chybami bude čtenář právem považovat za nedůvěryhodný.

            \item Výčet (v \LaTeX u lze sázet např. pomocí prostředí \texttt{itemize} nebo \texttt{enumerate}) ani obrázek nesmí začínat hned pod nadpisem a nesmí tvořit celou kapitolu.

            \item Nepoužívejte velké množství malých obrázků. Zvažte, zda je nelze seskupit.
        \end{itemize}






    \section{České odlišnosti}
        Česká sazba se oproti okolnímu světu v některých aspektech mírně liší.
        Jednou z odlišností je sazba uvozovek.
        Uvozovky se v češtině používají převážně pro zobrazení přímé řeči, zvýraznění přezdívek a ironie.
        V češtině se používají uvozovky typu \uv{9966} místo ''anglických uvozovek''.
        Lze je sázet připravenými příkazy nebo při použití UTF-8 kódování i přímo.

        Ve smíšené sazbě se řez uvozovek řídí řezem prvního uvozovaného slova.
        Pokud je uvozována celá věta, sází se koncová tečka před uvozovku, pokud se uvozuje slovo nebo část věty, sází se tečka za uvozovku.

        Druhou odlišností je pravidlo pro sázení konců řádků.
        V české sazbě do bloku by řádek neměl končit osamocenou jednopísmennou předložkou nebo spojkou.
        Spojkou \uv{a} končit může pouze při sazbě do šířky 25 liter.
        Abychom \LaTeX u zabránili v sázení osamocených předložek, spojujeme je s následujícím slovem \emph{nezlomitelnou mezerou}.
        Tu sázíme pomocí znaku \~{ } (vlnka, tilda).
        Pro systematické doplnění vlnek slouží volně šiřitelný program vlna od pana Olšáka\footnote[2]{Viz \href{http://petr.olsak.net/ftp/olsak/vlna/}{\url{http://petr.olsak.net/ftp/olsak/vlna/}}}
        , který je vhodné si v rámci projektu vyzkoušet.

        Balíček \texttt{fontenc} slouží ke korektnímu kódovaní znaků s diakritikou, aby bylo možno v textu vyhledávat a kopírovat z něj.


    \section{Závěr}
        Tento dokument schválně obsahuje několik typografických prohřešků.
        Sekce 2 a 3 obsahují typografické chyby.
        V kontextu celého textu je jistě snadno najdete.
        Je dobré znát možnosti \LaTeX u, ale je také nutné vědět, kdy je nepoužít.


\end{document}




Vysoké učení technické v BrněFakulta informačních technologiíTypografie a publikování –– 2. projektSazba dokumentů a matematických výrazů2024Libor Škarvada (iskarvada)
Úvod
V této úloze si vysázíme titulní stranu a kousek ma-tematického textu, v němž se vyskytují například De-finice 1 nebo rovnice (2) na straně 1. Pro vytvořenítěchto odkazů používáme kombinace příkazů\label,\ref,\eqrefa\pageref. Před odkazy patří nezlo-mitelná mezera. Pro zvýrazňování textu se používajípříkazy\verba\emph.Titulní strana je vysázena prostředímtitlepagea nadpis je v optickém středu s využitímpřesnéhozla-tého řezu, který byl probrán na přednášce. Dále jsouna titulní straně čtyři různé velikosti písma a mezidvojicemi řádků textu je použito řádkování se zada-nou relativní velikostí 0,5 em a 0,6 em1.1  Matematický textMatematické symboly a výrazy v plynulém textu jsouv prostředímath. Definice a věty sázíme v prostředídefinovaném příkazem\newtheoremz balíkuamsthm.Tato prostředí obracejí význam\emph: uvnitř textusázeného kurzívou se zvýrazňuje písmem v základ-ním řezu. Někdy je vhodné použít konstrukci${}$nebo\mbox{}, která zabrání zalomení (matematic-kého) textu. Pozor také na tvar i sklon řeckých písmen:srovnejte\epsilona\varepsilon,\Xia\varXi.Definice 1.Konečný přepisovací strojneboliMea-lyho automatje definován jako uspořádaná pěticetvaruM= (Q,Σ,Γ,δ,q0), kde:•Qje konečná množinastavů,•Σje konečnávstupní abeceda,•Γje konečnávýstupní abeceda,•δ:Q×Σ→Q×Γje totálnípřechodová funkce,•q0∈Qjepočáteční stav.1.1  Podsekce s definicíPomocí přechodové funkceδzavedeme novou funkciδ∗pro překlad vstupních slovu∈Σ∗do výstupních slovw∈Γ∗.Definice 2.NechťM= (Q,Σ,Γ,δ,q0)je Mealyhoautomat.Překládací funkceδ∗:Q×Σ∗×Γ∗→Γ∗je pro každý stavq∈Q, symbolx∈Σ, slovau∈Σ∗,w∈Γ∗definována rekurentním předpisem:•δ∗(q,ε,w) =w•δ∗(q,xu,w) =δ∗(q′,u,wy), kde(q′,y) =δ(q,x)1Použijte správný typ mezery mezi číslem a jednotkou.1.2  RovniceSložitější matematické formule sázíme mimo plynulýtext pomocí prostředídisplaymath. Lze umístit i vícevýrazů na jeden řádek, ale pak je třeba tyto vhodněoddělit, například pomocí\quad, při dostatku místai\qquad.gan/∈ABny10−5√x+7√y   x > y2≥y3Velikost závorek a svislých čar je potřeba přizpů-sobit jejich obsahu. Velikost lze stanovit explicitně,anebo pomocí\lefta\right. Kombinační čísla sá-zejte makrem\binom.∣∣∣⋃P∣∣∣=∑∅6=X⊆P(−1)|X|−1∣∣∣⋂X∣∣∣Fn+1=(n0)+(n−11)+(n−22)+···+(⌈n2⌉⌊n2⌋)V rovnici (1) jsou tři typy závorek s různouexpli-citnědefinovanou velikostí. Obě rovnice mají svisle za-rovnaná rovnítka. Použijte k tomu vhodné prostředí.({b⊗[c1⊕c2]◦a}23)=  logzx(1)∫baf(x) dx=−∫abf(y) dy(2)V této větě vidíme, jak se vysází proměnná určujícílimitu v běžném textu:limm→∞f(m). Podobně je toi s dalšími symboly jako⋃N∈MNči∑mi=1x2i. S vynu-cením méně úsporné sazby příkazem\limitsbudouvzorce vysázeny v podobělimm→∞f(m)am∑i=1x2i.2  MaticePro sázení matic se používá prostředíarraya závorkys výškou nastavenou pomocí\left,\right.D=∣∣∣∣∣∣∣∣∣∣a11a12···a1na21a22···a2n............am1am2···amn∣∣∣∣∣∣∣∣∣∣=∣∣∣∣∣x  yt  w∣∣∣∣∣=xw−ytProstředíarraylze úspěšně využít i jinde, napříkladna pravé straně následující rovnosti.(nk)={n!k!(n−k)!pro0≤k≤n0jinak1

