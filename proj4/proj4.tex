%preambula
\documentclass[11pt]{article}

%packages
\usepackage[a4paper,
            total={17cm, 24cm},
            left=2cm,
            top=3cm,]{geometry}
\usepackage{babel}[czech]
\usepackage{times}
\usepackage[hidelinks]{hyperref}
\usepackage[utf8]{inputenc}
\usepackage[T1]{fontenc}



%document
\begin{document}

    \begin{titlepage}
        \begin{center}
            \Huge\textsc{Vysoké učení technické v Brně}\\
            \huge\textsc{Fakulta informačních technologií}\\
            \vspace{\stretch{0.382}}
            {Typografie a publikování –– 4. projekt}\\

            \vspace{\stretch{0.618}}
        \end{center}
        {\LARGE 4. dubna 2024 \hfill
        Katarína Mečiarová (xmeciak00)}

    \end{titlepage}

    Vaším úkolem je vytvořit smysluplný konzistentní krátký (cca 20-30 vět) dokument o tématu z oblasti typografie (pokud vyčerpáte literaturu na téma typografie, můžete psát i o informatice, fyzice nebo matematice, ovšem dodržte minimální počet citovaných publikací jednotlivých typů), ve kterém budete citovat publikace následujících typů:

    2 monografie (alespoň jedna musí být cizojazyčná),
    3 elektronické dokumenty (online),
    1 seriálovou publikaci (celý tištěný časopis či sborník z konference),
    2 články v seriálové publikaci (zahraniční tištěné články),
    2 kvalifikační práce (bakalářské, diplomové nebo disertační).

    Seznam použité literatury na konci dokumentu vysázejte nástrojem BiBTeX a použijte takový styl, který nejvíce odpovídá české normě. Citujte dle pokynů uvedených na přednášce a v souladu s normou ČSN ISO 690. Přímá citace může být maximálně jedna, zbytek budou parafráze. Internetové (např. citace ze zive.cz, atd.) dokumenty mohou být pouze tři, ostatní musí být primárně tištěné. Na hodnocení bude mít vliv především způsob a přesnost citací, ale také způsob odkazování na literaturu v samotném textu. Na výsledné hodnocení budou mít ale také vliv případné pravopisné chyby a nedodržení obecných typografických zásad. S dokumentem odevzdejte také bib soubor a příslušný makefile. Kontrola bude opět na Merlinovi. Odevzdejte všechny použité soubory tak, aby šel projekt přelozit. Nepřeložitelné projekty budou hodnoceny 0 body. Jako úvodní stránku použijte třeba tu ze třetího projektu. Komprimovany soubor s odevzdanymi soubory pojmenujte svym loginem (nedávejte soubory do adresáře). Projekt zadává, konzultuje a hodnotí Jaroslav Rozman.



\end{document}