% Preamble
\documentclass[a4paper, 11pt, twocolumn]{article}[25.2.2024]

% Packages
\usepackage[top=1.8cm, inner=1.8cm, outer=1cm]{geometry}
\usepackage[utf8]{inputenc}
\usepackage[T1]{fontenc}
\usepackage[czech]{babel}
\usepackage{alltt}
\usepackage{changepage}
\usepackage{enumitem}
\usepackage{setspace}
\usepackage{amsmath}
\usepackage{amsfonts}
\usepackage{amssymb}
\usepackage[version=4]{mhchem}
\usepackage{stmaryrd}


%document
\begin{document}
    %title
    \begin{titlepage}
        \begin{center}
            \Huge\textsc{Vysoké učení technické v Brně}\\
            \huge\textsc{Fakulta informačních technologií}\\
            \vspace{\stretch{0.382}}
            {Typografie a publikování –– 2. projekt}\\
                Sazba dokumentů a matematických výrazů\\
            \vspace{\stretch{0.618}}
        \end{center}
        {\LARGE 2024 \hfill
        Katarína Mečiarová (xmeciak00)}
     
\end{titlepage}



\section*{Úvod}
V této úloze si vysázíme titulní stranu a kousek matematického textu, v němž se vyskytují například Definice 1 nebo rovnice (2) na straně 1 . Pro vytvoření těchto odkazů používáme kombinace příkazů \verb|\label|, \verb|\ref|, \verb|\eqref|a \verb|\pageref|. Před odkazy patří nezlomitelná mezera. Pro zvýrazňování textu se používají př́kazy \verb|\verb| a \verb|\emph|.

Titulní strana je vysázena prostředím \verb|titlepage| a nadpis je v optickém středu s využitím \emph{přesného} zlatého řezu, který byl probrán na přednášce. Dále jsou na titulní straně čtyři různé velikosti písma a mezi dvojicemi řádkư textu je použito řádkování se zadanou relativní velikostí $0,5 \mathrm{em}$ a $0,6 \mathrm{em}^{1}$.

\section*{1 Matematický text}
Matematické symboly a výrazy v plynulém textu jsou v prostředí \verb|math|. Definice a věty sázíme v prostředí definovaném příkazem \verb|\newtheorem| z balíku \verb|amsthm|. Tato prostředí obracejí význam \verb|\emph|: uvnitř textu sázeného kurzívou se zvýrazňuje písmem v základním řezu. Někdy je vhodné použít konstrukci \verb|${}$| nebo \verb|\mbox{}|, která zabrání zalomení (matematického) textu. Pozor také na tvar i sklon řeckých písmen: srovnejte \verb|\epsilon| a \verb|\varepsilon|, \verb|\Xi| a \verb|\varXi|.\\

\newtheorem{Definice 1.}{\emph{Konečný přepisovací stroj} neboli \emph{Mealyho automat} je definován jako uspoŕádaná pětice tvaru M = (Q, Σ, Γ, δ, q0), kde:}

\begin{itemize}
  \item Q je konečná množina stavů,
  \item $\Sigma$ je konečná vstupní abeceda,
  \item $\Gamma$ je konečná výstupní abeceda,
  \item $\delta: Q \times \Sigma \rightarrow Q \times \Gamma$ je totální přechodová funkce,
  \item $q_{0} \in Q$ je počáteční stav.
\end{itemize}

\subsection*{1.1 Podsekce s definicí}
Pomocí přechodové funkce $\delta$ zavedeme novou funkci $\delta^{*}$ pro překlad vstupních slov $u \in \Sigma^{*}$ do výstupních slov $w \in \Gamma^{*}$.

Definice 2. Necht $M=\left(Q, \Sigma, \Gamma, \delta, q_{0}\right)$ je Mealyho automat. Překládací funkce $\delta^{*}: Q \times \Sigma^{*} \times \Gamma^{*} \rightarrow \Gamma^{*}$ je pro každý stav $q \in Q$, symbol $x \in \Sigma$, slova $u \in \Sigma^{*}$, $w \in \Gamma^{*}$ definována rekurentním předpisem:\\
- $\delta^{*}(q, \varepsilon, w)=w$\\
- $\delta^{*}(q, x u, w)=\delta^{*}\left(q^{\prime}, u, w y\right), k d e\left(q^{\prime}, y\right)=\delta(q, x)$
\footnotetext{${ }^{1}$ Použijte správný typ mezery mezi číslem a jednotkou.
}

\subsection*{1.2 Rovnice}
Složitější matematické formule sázíme mimo plynulý text pomocí prostředí displaymath. Lze umístit i více výrazů na jeden řádek, ale pak je třeba tyto vhodně oddělit, například pomocí \textbackslash quad, při dostatku místa i \textbackslash qquad.

$$
g^{a_{n}} \notin A^{B^{n}} \quad y_{0}^{1}-\sqrt[5]{x+\sqrt[7]{y}} \quad x>y^{2} \geq y^{3}
$$

Velikost závorek a svislých čar je potřeba přizpůsobit jejich obsahu. Velikost lze stanovit explicitně, anebo pomocí \textbackslash left a \textbackslash right. Kombinační čísla sázejte makrem \textbackslash binom.

$$
|\bigcup P|=\sum_{\emptyset \neq X \subseteq P}(-1)^{|X|-1}|\bigcap X|
$$

$$
F_{n+1}=\left(\begin{array}{c}
n \\
0
\end{array}\right)+\left(\begin{array}{c}
n-1 \\
1
\end{array}\right)+\left(\begin{array}{c}
n-2 \\
2
\end{array}\right)+\cdots+\left(\begin{array}{c}
\left\lceil\frac{n}{2}\right\rceil \\
\left\lfloor\frac{n}{2}\right\rfloor
\end{array}\right)
$$

V rovnici (1) jsou tři typy závorek s rưznou explicitně definovanou velikostí. Obě rovnice mají svisle zarovnaná rovnítka. Použijte k tomu vhodné prostředí.


\begin{align*}
\left(\left\{b \otimes\left[c_{1} \oplus c_{2}\right] \circ a\right\}^{\frac{2}{3}}\right) & =\log _{z} x  \tag{1}\\
\int_{a}^{b} f(x) \mathrm{d} x & =-\int_{b}^{a} f(y) \mathrm{d} y \tag{2}
\end{align*}


V této větě vidíme, jak se vysází proměnná určující limitu v běžném textu: $\lim _{m \rightarrow \infty} f(m)$. Podobně je to i s dalšími symboly jako $\bigcup_{N \in \mathcal{M}} N$ či $\sum_{i=1}^{m} x_{i}^{2}$. S vynucením méně úsporné sazby příkazem \textbackslash limits budou vzorce vysázeny v podobě $\lim _{m \rightarrow \infty} f(m)$ a $\sum_{i=1}^{m} x_{i}^{2}$.

\section*{2 Matice}
Pro sázení matic se používá prostředí array a závorky s výškou nastavenou pomocí \textbackslash left, \textbackslash right.

$$
D=\left|\begin{array}{cccc}
a_{11} & a_{12} & \cdots & a_{1 n} \\
a_{21} & a_{22} & \cdots & a_{2 n} \\
\vdots & \vdots & \ddots & \vdots \\
a_{m 1} & a_{m 2} & \cdots & a_{m n}
\end{array}\right|=\left|\begin{array}{cc}
x & y \\
t & w
\end{array}\right|=x w-y t
$$

Prostředí array lze úspěšně využít i jinde, například na pravé straně následující rovnosti.

$$
\left(\begin{array}{l}
n \\
k
\end{array}\right)= \begin{cases}\frac{n !}{k !(n-k) !} & \text { pro } 0 \leq k \leq n \\
0 & \text { jinak }\end{cases}
$$


\end{document}